\chapter{Project defence}
\label{ch:project_defence}

In what follows the implementation details of the assignment are discussed.
Multiple figures were made to illustrate the working of different parts of the code.
A legend showing the meaning of the parts of the figure is given in figure \ref{fig:legend}.
For more technical details the reader is invited to read the comments in the code.

%------------------------------------

\section{Main application structure}
\label{sec:main_app_structure}

There are two main runnable applications: \texttt{Main} and \texttt{MainDisplay}. 
Both extend the App object and run a \texttt{RunnableGraph}.
They differ in the way they handle output by having differing sinks.
\texttt{Main} saves the output to text files in the output folder whilst \texttt{MainDisplay} prints the output to the terminal.
The former is the assignment's requirement and the latter is an easier to debug alternative.
A simplified representation of the Main application is given in figure \ref{fig:main}.
The different components will be explained in greater detail in what follows.

The most important remark to make for the main graph is the use of \texttt{alsoToMat} to reach two distinct sinks.
Alternatively, a broadcast mechanism could have been used.
These multiple sinks are required for the second term extension of the assignment, each saving a differing file.
The discussed \textit{bug} for which the teaching assistants help was received was simply the use of \texttt{to} instead of \texttt{toMat} for flowing to the sink, causing premature termination of the graph.  

\begin{figure}[H]
    \centering
    \includegraphics[width=0.8\linewidth]{images/main.png}
    \captionsetup{width=0.7\linewidth}
    \captionsetup{justification=centering}
    \caption{Simplified representation of the main application.  }
    \label{fig:main}
\end{figure}





%------------------------------------

\section{MD, MDC and MDS objects}
\label{sec:MD_vs_MDC_vs_MDS}

There are a total of three case classes for this assignment.
\texttt{MavenDependency} (\texttt{MD}) represents a singular Maven dependency as instantiated from the extracted file.
It has a \texttt{library} (name), \texttt{dependency} and \texttt{dependency\_type}. 
Thus multiple dependencies of an equal Maven library give rise to different \texttt{MD} objects.
Its companion object provides an alternative constructor.

\texttt{MavenDependencyCount} (\texttt{MDC}) represents a Maven library and its dependencies.
It has a \texttt{library} (name) and a dependency count per dependency type (\texttt{compile}, \texttt{provided}, \texttt{runtime} and \texttt{test}).
Its companion object provides a function for combining two \texttt{MDC} objects.
The \texttt{MavenDependencyStatistics} (\texttt{MDS}) object is used for the statistics part of the assignment.
It has a \texttt{minimumDependencies} and a count of libraries eligible for the statistic per dependency type.
Its companion object provides a function for adding a \texttt{MDC} object to the statistic object.


%------------------------------------

\section{Source generation process}
\label{sec:source_gen_proc}

Figure \ref{fig:main} shows how \texttt{Main} makes use of a source generation process to provide a list of \texttt{MD} objects as source.
This source is available as \texttt{sourceOfMavenDependency} under the \texttt{Sources} object.
The process of generating this source is visualised in figure \ref{fig:source}.

The process makes use of an initial \texttt{ByteString} source which is simply the extracted file.
The raw \texttt{ByteString} source is then parsed as a CSV through \texttt{CsvParsing} whose records are then converted to a map of strings.
This map can be used to instantiate \texttt{MD} objects resulting in a source of \texttt{MD} objects.

\begin{figure}[H]
    \centering
    \includegraphics[width=0.85\linewidth]{images/source.png}
    \captionsetup{width=0.8\linewidth}
    \captionsetup{justification=centering}
    \caption{Simplified representation of the source generation process.  }
    \label{fig:source}
\end{figure}




%------------------------------------

\section{Source generation process}
\label{sec:source_gen_proc}

Figure \ref{fig:main} shows how \texttt{Main} makes use of the explained \texttt{MD} object source to collect \texttt{MDC} objects through a dependency count process.
This dependency count process is available as \texttt{flowMavenDependencySource\-ToCollectedMavenDependencyCount} under the \texttt{Flows} object.
This flow makes use of custom flow shapes and is visualised in figure \ref{fig:counting}.

First, the incoming stream of \texttt{MD} objects is grouped on their library name to create a substream for each library.
\texttt{flowMavenDependency\-ToMavenDependencyCount\-Parallel} is a custom flow shape responsible for making a balanced approach of counting the dependencies by adding a \texttt{Balance\-[MavenDependency]} having two ports.
These two ports then perform the same custom flow shape \texttt{flowMavenDependencyToMavenDependencyCount} to perform the actual counting and are merged on the end.
All of this is included in \texttt{flow\-Maven\-Dependency\-To\-Maven\-Dependency\-Count\-Parallel}.

\texttt{flowMavenDependencyToMavenDependencyCount} is shown in detail.
The counting happens by broadcasting the incoming \texttt{MD} objects to four separate flows.
These flows generate \texttt{MDC} objects having either a count of one or zero for the field they are counting.
All of these \texttt{MDC} objects are then merged to a singular \texttt{MDC} object by using \texttt{toSingleMavenDependencyCount}, which uses \texttt{flowMultipleMavenDependencyCountsToSingle} which in it turns uses \texttt{merge\-Maven\-Dependency\-Count} from the \texttt{MDC} companion object. 

\begin{figure}[H]
    \centering
    \includegraphics[width=0.95\linewidth]{images/counting.png}
    \captionsetup{width=0.7\linewidth}
    \captionsetup{justification=centering}
    \caption{Simplified representation of the counting process.  }
    \label{fig:counting}
\end{figure}




%------------------------------------

\section{Saving process}
\label{sec:sink_proc}

Figure \ref{fig:main} shows how \texttt{Main} makes use of the explained \texttt{MDC} objects to save dependency count and statistical information to a text file.
This saving process is available as the \texttt{textual\-Maven\-Dependency\-Count\-Save\-Sink} and \texttt{textualMavenDependencyStatisticsSaveSink} sink under the \texttt{Sinks} object.
Figure \ref{fig:counting} visualises the saving process.

Saving consists of generating the right \texttt{ByteString} stream which can then be saved by \texttt{FileIO.toPath}.
For the first sink, this just consists of converting each input \texttt{MDC} object to a textual representation in the specified form.
For the second sink, the input \texttt{MDC} objects need conversion to \texttt{MDS} objects first before the specified textual representation can be made.
In order to do this, the function \texttt{mergeMavenDependencyCount} from the companion object of the \texttt{MDS} class can be used in a folding manner.
This is done in \texttt{flow\-MavenDependency\-Count\-To\-Maven\-Dependency\-Statistics} where \texttt{minimumDependencies} can also be specified.


\begin{figure}[H]
    \centering
    \includegraphics[width=0.95\linewidth]{images/sinks.png}
    \captionsetup{width=0.7\linewidth}
    \captionsetup{justification=centering}
    \caption{Simplified representation of the saving process. }
    \label{fig:sinks}
\end{figure}

\begin{figure}[H]
    \centering
    \includegraphics[width=0.95\linewidth]{images/legend.png}
    \captionsetup{width=0.7\linewidth}
    \captionsetup{justification=centering}
    \caption{Legend for the figures. }
    \label{fig:legend}
\end{figure}